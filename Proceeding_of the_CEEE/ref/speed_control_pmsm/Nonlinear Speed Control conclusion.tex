\documentclass[onecolumn]{IEEEtranTIE}
%\documentclass[journal]{IEEEtranTIE}
\usepackage{graphicx}
\usepackage{cite}
\usepackage{picinpar}
\usepackage{amsmath}
\usepackage{url}
\usepackage{flushend}
\usepackage[latin1]{inputenc}
\usepackage{colortbl}
\usepackage{soul}
\usepackage{multirow}
\usepackage{pifont}
\usepackage{color}
\usepackage{alltt}
\usepackage[hidelinks]{hyperref}
\usepackage{enumerate}
\usepackage{siunitx}
\usepackage{breakurl}
\usepackage{epstopdf}
\usepackage{pbox}
\usepackage{psfrag}
\usepackage{makecell}
\usepackage{bm}
\usepackage{subfigure}

\begin{document}
\title{	Response to Reviewers }
\maketitle

\newsavebox\newreach
\begin{lrbox}{\newreach}
\begin{minipage}{0.1\textwidth}
\begin{align}
\dot{s} &= -eq(x_1,s) sgn(s)\nonumber\\
eq(x_1,s) &= \frac{k}{\epsilon + (1+\frac{1}{\left|{x_1}\right|}-\epsilon)e^{-\delta \left|{s}\right|}}\nonumber
\end{align}
\end{minipage}
\end{lrbox}


\newsavebox\unistress
\begin{lrbox}{\unistress}
  \begin{minipage}{0.1\textwidth}
\begin{align}
\frac{d i_d}{d t} &= \frac{u_{d}^*}{L} + d_{d} \nonumber\\
\frac{d i_q}{d t} &= \frac{u_{q}^*}{L} + d_{q} \nonumber
\end{align}
 \end{minipage}
\end{lrbox}



\begin{table}[!t]
	\renewcommand{\arraystretch}{1.3}
	\caption{COMPARISON OF Equal reaching law and Novel reaching law}  %表格的名字
	\centering
	\label{comparison9andpaper}
	%\centering
	\resizebox{\columnwidth}{!}{
	\begin{tabular}{{l  l  l}}  % 表格开始,两个C表示分为两列
			\hline\hline \\[-3mm]
Descriptions & Equal reaching law & Novel reaching law\\    %表头
\hline     %一道分隔线
reaching law          & $\dot{s} = -k_1 \cdot sgn(s)$ & \usebox{\newreach}\\
a faster reaching time           &  $k_1$    					  & $\frac{k}{\epsilon}$  \\   %数据
chattering             & $  $                  &  $eq(x_1,s)$ gradually decreases to zero to suppress the chattering \\
reaching time t						 &	$t_1 =\frac{ \left|s(0) \right|}{k_1}$						& $t < \frac{\epsilon \left|s(0) \right|}{k}$\\
reduce the chattering of SMC            & none      					  & $k < \epsilon k_1$  \\   %数据
bandwidth					 & $\Delta = k_1 T$																													& $\Delta \approx \frac{k \left| x_1 \right| T}{1+\left| x_1 \right|}$\\
		[1.4ex] \hline\hline
		\end{tabular}
}
\end{table}



\end{document}
